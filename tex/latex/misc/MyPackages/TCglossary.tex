\newacronym{tpack}{TPACK}
{Technological Pedagogical and Content Knowledge}

\newacronym{WHO}{WHO}
{World Health Organization}

\newacronym{ADA}{ADA}
{American with Disabilities Act}

\newacronym{asl}{ASL}
{American Sign Language}

\newacronym{ud}{UD}
{Universal Design}


\newacronym{mmag}{MMAG}
{Mathematical modeling: Applications with {G}eo{G}ebra}

\newglossaryentry{latex environment}
{
	name={\LaTeX\ environment},
	description={A way for \LaTeX to delineate parts of a document that need to be treated with special care. Environments always has a \textbackslash begin \{...\} and \textbackslash end \{...\} }
}

\newglossaryentry{thick description}
{
	name={thick description},
	description={A description that includes not only immediate behaviors but also contextual and experimental understandings}
}


\newacronym{MKT}{MKT}
{Mathematical Knowledge for Teachers}

\newacronym{CCK}{CCK}
{Common Content Knowledge}

\newacronym{SCK}{SCK}
{Specific Content Knowledge}

\newacronym{KCS}{KCS}
{Knowledge of Content and Students}

\newacronym{KCT}{KCT}
{Knowledge of Content and Teaching}


\newacronym{ssd}{SSD}
{Sensory Substitution Device}


\newglossaryentry{mixed reality}
{
	name={Mixed Reality},
	description={A mixed reality experience is one that seamlessly blends the user’s real-world environment and digitally-created content, where both environments can coexist and interact with each other \parencite{noauthor_what_2019}}
}

\newacronym{mr}{MR}
{\gls{mixed reality}}


\newglossaryentry{extended reality}
{
	name={Extended Reality},
	description={An umbrella term that includes \acrlong{vr}, \acrlong{ar} and \acrlong{mr}}
}

\newacronym{xr}{XR}
{\gls{extended reality}}

\newacronym{DCLM}{DCLM}
{Divergent Collaboration Learning Mechanisms}

\newacronym{rtwh}{RTWH}
{Reading, Thinking and Writing about History: teaching argument writing to diverse learners in the common core classroom, grades 6-12}

\newacronym{hpl}{HPL}
{How people learn}

\newacronym{CDA}{CDA}
{Critical Discourse Analysis}

\newglossaryentry{differentiated scaffolding}
{
	name={differentiated scaffolding},
	description={Use different means to support different aspect of performance}
}

\newglossaryentry{redundant scaffolding}
{
	name={redundant scaffolding},
	description={the same aspect of performance is supported through different means}
}

\newglossaryentry{synergistic scaffolding}
{
	name={synergistic scaffolding},
	description={providing different concurrent means of support for the same performance that are synergistic in scaffolding learners. In the design strategy, some of the supports help the learners make use of the other concurrent supports }
}
\longnewglossaryentry{naturalistic observation}
{
	name={naturalistic observation},
	description={
Non-experimental method where people/groups/organization/cultures are observed in their "natural" setting. (Quotes because by being in it you might change the setting somewhat.)


\begin{tabular}{p{1in}|p{2in}|p{2in}}
& Participant & Non-participant \\ \hline
Overt (participant aware) & 
Living as a member of a community & 
Researcher not mingling (like in a classroom) 
Challenging to be a "fly on the wall"
\\ \hline
Covert (participant not aware) & 
IRB needs to know why you're not making it overt & 
\end{tabular}
}
}

\longnewglossaryentry{distributed scaffolding}
{
	name={distributed scaffolding},
	description={A concept that describes those cases in which scaffolding is embedded productively in multiple aspects of a learning environment.  \parencite{reiser_scaffolding_2014} Three types:
	\begin{enumerate}
	\item \gls{differentiated scaffolding} 
	\item \gls{redundant scaffolding} 
	\item \gls{synergistic scaffolding}
	\end{enumerate}
	}
}

\newglossaryentry{prolepsis}
{
	name={prolepsis},
	description={A mechanism that helps explain how learning occurs through scaffolding, where the learner imitates the modeled actions and associates them with the directive and definition of the task \parencite{reiser_scaffolding_2014}}
}


\newglossaryentry{gls-ZPD} 
{
	name={Zone of Proximal Development},
	description={A range of tasks that are outside of the learners' independent ability but are achievable with appropriate help, thereby extending their range of indpendent activity \parencite{reiser_scaffolding_2014}},
}

\newacronym{zpd}{ZPD}{\Gls{gls-ZPD}}


\newacronym{ec}{EC}
{Epistemic Cognition}

\newacronym{GSC}{GSC}
{Graduate Student Council}

\newacronym{LSSA}{LSSA}
{Learning Sciences Student Association}

\newacronym{LSGSC}{LSGSC}
{Learning Sciences Graduate Student Conference}

\newacronym{AJE}{AJE}
{American Journal of Education}

\newacronym{esp}{ESP}
{Emerging Scholars Program}

\newacronym{stem}{STEM}
{Science, Technology, Engineering and Mathematics}

\newacronym{PtC}{PtC}
{Progress through Calculus}

\newacronym{SEMINAL}{SEMINAL}
{Student Engagement in Mathematics through an Institutional Network for Active Learning}

\newacronym{FINER}{FINER}
{Criterons for a Good Research Question: Feasible, Interesting, Novel, Ethical and Relevant}

\newacronym{PICOT}{PICOT}
{Components of a research question: Population, Investigation, Comparison Group, Outcome of Interest and Time}

\newacronym{rct}{RCT}
{Randomized Controlled Trials}

\newacronym{dbr}{DBR}
{Design Based Research}

\newacronym{IES}{IES}
{Institute of Education Sciences}

\newacronym{lpp}{LPP}
{Legitimate Peripheral Participation}

\newacronym{udl}{UDL}
{Universal Design for Learning} 

\newacronym{gbl}{GBL}
{Game Based Learning}

\newglossaryentry{virtual reality}
{
	name={Virtual Reality},
	description={Computer-generated stereo visuals which entirely surround the user, entirely replacing the real world environment around them. Real-time user interaction within the virtual environment is possible, whether through detailed interactions, or simply being able to look around within the experience \parencite{noauthor_what_2019} }
}

\newglossaryentry{augmented reality}
{
	name={Augmented Reality},
	description={Augmented reality is the overlaying of digitally-created content on top of the real world. Augmented reality - or 'AR' – allows the user to interact with both the real world and digital elements or augmentations \parencite{noauthor_what_2019} }
}

\newacronym{vrar}{VR/AR}
{\gls{virtual reality} and \gls{augmented reality}}


\newacronym{vr}{VR}
{\gls{virtual reality}}


\newacronym{ar}{AR}
{\gls{augmented reality}}



\newacronym{ltp}{LT/P}
{Learning Trajectories/Progression}

\newacronym{AERA}{AERA}
{American Educational Research Association}

\newacronym{ICLS}{ICLS}{International Conference of the Learning Sciences}

\newacronym{CSCL}{CSCL}
{computer support for collaborative learning}

\newacronym{ACM}{ACM}
{Association for Computing Machinery} 

\newacronym{AACE}{AACE}
{Association for the Advancement of Computing in Education}

\newacronym{CSCW}{CSCW}
{Computer-Supported Cooperative Work conference}

\newacronym{IRL}{IRL}
{The Institute for Research on Learning}

 
\newglossaryentry{externalist}
{
	name=externalist,
	description={External accounts view the properties of the environment as the principal factors explaining the properties of the mind \parencite{bredo_philosophies_2006}}
}


\newglossaryentry{internalist}
{
	name=internalist,
	description={Internalist accounts suggest that the most important determinants of thought or knowledge arise from the "inner" constraints of the mind or distinctions built into language or culture \parencite{bredo_philosophies_2006}}
}


\newglossaryentry{interactionism}
{
	name=interactionism,
	description={Thinking alters action, which subsequently affects the external world, thereby affecting one's future sensory input; i.e., "internal" and "external" factors affect one another \parencite{bredo_philosophies_2006}}
}


\newglossaryentry{empiricism}
{
	name=empiricism,
	description={Empiricist philosophies of knowledge generally argue that knowledge is based in experience of concrete objects or events; the philosophy that all knowledge comes from "experience". (Primary vs secondary qualities: primary = objective; e.g., solidity/shape/size/velocity and secondary = subjective; e.g., taste/color) \parencite{bredo_philosophies_2006}}
}


\newglossaryentry{classicalpositivism}
{
	name={classical positivism},
	description={The view that knowledge should be based on what is "positively" and directly observed rather than on unobserved entities, forces, or causes thought to lie behind things \parencite{bredo_philosophies_2006}}
}


\newglossaryentry{logicalpositivism}
{
	name={logical positivism},
	description={A descendant philosophy to \gls{classicalpositivism}, where the use of general form of logic is adopted, and sensitivity to analyzing the language in which scientific propositions \parencite{bredo_philosophies_2006}}
}


\longnewglossaryentry{postpositivism}
{
	name={post positivism},
}
{Critiques of \gls{logicalpositivism}:
\begin{itemize}
\item the notion that observational facts are logically independent of the theories they test;
\item the notion that scientific thought can be "verified" is questionable (This has been called "\gls{criticalrationalism}"),
\item positivist conception of science can be overly individualistic, 
\item the attempts to separate facts from values has failed, in that there is no value-neutral language, and one's purposes or aims are effectively built into the way one conceives things.
\end{itemize}
This is not necessarily a comprehensive list of critiques of positivist thoughts \parencite{bredo_philosophies_2006}
}


\newglossaryentry{criticalrationalism}
{
	name={critical rationalism},
	description={The notion that what makes a claim "scientific" is the fact that it is at least \emph{potentially falsifiable} by empirical observations and has been subject to stringent attempts at such disproof \parencite{bredo_philosophies_2006}}	
}

\newglossaryentry{rationalism}
{
	name=rationalism,
	description={Rationalists view knowledge as primarily determined by reason, or the mind, rather than by sensory experience}
}


\newglossaryentry{hermeneutics}
{
	name={hermeneutics},
	description={The science of textual interpretation origially develped as a systemic method of interpretation to resolve disputes over religious texts. Modern interpretation of this is that the author of an act or utterance does not have privileged access to its meaning and must rely on the responses of others to clarify it.  \parencite{bredo_philosophies_2006}}
}


\newglossaryentry{subjectiveidealism}
{
	name={subjective idealism},
	description={NOT WELL DEFINED Rationalism? Hermaneutics? \parencite{bredo_philosophies_2006}}
}


\newglossaryentry{structuralism}
{
	name={structuralism},
	description={Per Roman Jacobson, "any set of phenomena examined by contemporary science [which are] treated not as a mechanical agglomeration but as a structural whole (in which) the mechanical conception of processes yields to the question of their function". The elements or objects figuring in the analysis of behavior are defined by the rules constituting the activity of which that behavior is a part.  \parencite{bredo_philosophies_2006}}
}


\newglossaryentry{postmodernism}
{
	name={postmodernism},
	description={Lyotard defined this as "incredulity toward meta-narratives", which are any account that attempts to encompass other accounts to become \emph{the} way things are. In this, the postmodernists reject any attempt to give a unifying theory or totalizing accounts of the human condition.  \parencite{bredo_philosophies_2006}}
}


\newglossaryentry{dialectic}
{
	name={dialectic},
	description={logical argument in collective thought \parencite{bredo_philosophies_2006}}
}

\newglossaryentry{absoluteidealism}
{
	name={absolute idealism},
	description={Hegel (1830-1991) philosophy, where he believes that inquiry functions best when conducted with an awareness that it is just a "moment" in a collective, evolutionary process. This is a type of anti-dogmatism. \parencite{bredo_philosophies_2006}}
}


\newglossaryentry{dialecticalmaterialism}
{
	name={dialectical materialism},
	description={NOT WELL DEFINED (Marx??) \parencite{bredo_philosophies_2006}}
} 
 
 
\newglossaryentry{criticaltheory}
{
	name={critical theory},
	description={Response to \gls{instrumentalrationality}, where a researcher may accept a supposedly universal conception of things that in practice serves only limited private interests. This line of thought by Habermas further claims that the validity of an assertion can only be determined in free and unforced argument in an "ideal speech situation" with "no external constraints preventing participants from assessing evidence and argument, and ...each participant has an equal and open chance of entering into discussion". \parencite{bredo_philosophies_2006}}
}

\newglossaryentry{instrumentalrationality}
{
	name={instrumental rationality},
	description={This is an applied form of utilitarism in Education and other social sciences, where one approaches everything in terms of efficiency towards some stated goals by assuming that they're shared by all, and thereby hiding or displacing the discussion on the conception of said goals. \parencite{bredo_philosophies_2006}}
}

\newglossaryentry{pragmatism}
{
	name={pragmatism},
	description={This line of thought do not believe that there is an ultimate "ideal" or "goal", and rebelled against all deterministic schemes and fixed schemes and foundations\parencite{bredo_philosophies_2006}}
}


\newglossaryentry{intermediary}
{
	name={intermediary},
	description={Intermediary is a "gatekeeper" who are often guardians,  and acts as a link between the researcher and the participant}
}

\newglossaryentry{member checking}
{
	name={member checking},
	description={The act of verifying with the interviewee that the data is accurate}
}

\newglossaryentry{reliability}
{
	name={reliability},
	description={Quantitative: Can be repeated, and is \emph{consistent}, with generalization in mind}
}

\newglossaryentry{validity}
{
	name={validity},
	description={How valid one's research is. \emph{Accurate} and reflects the real-world}
}